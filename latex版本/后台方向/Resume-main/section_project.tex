%======================================================================
\sectionTitle{项目经历}{\faListUl}
%======================================================================
\begin{experiences}
    \experience%
    [2017.10]
    {2018.9}%
    {\textbf{核心成员 @ 哨兵机器人}}%
    [
    全国大学生机器人大赛参赛作品,自动识别并跟踪敌方机器人的装甲板并进行精准射击。
        \begin{itemize}
            \item {编写哨兵嵌入式软件,包括运动射击策略、底层驱动、系统通信等工作;}
            \item {哨兵电路设计与调试,并负责传感器系统设计;}
            \item {编写串级PID算法控制电机并调参,控制精度在0.05度左右;}
            \item {编写卡尔曼滤波算法,融合视觉系统传递过来的敌方装甲板位置信息与云台当前位置信息,使哨兵云台能够平稳运动、自动瞄准并跟踪敌方装甲板,运动目标击中率达到30\%;}
        \end{itemize}
    ]
    [C/C++, STM32, kalman filter, PID, Git]
    \separator{0.5ex}

    \experience%
    [2020.10]
    {2021.6}%
    {\textbf{项目负责人 @ 基于EMI信号的电源质量监测系统设计 }}%
    [
    在电子系统电源主干路上采集并分类电磁干扰信号(EMI),对支线电路板的工作状态进行监测。
        \begin{itemize}
            \item {搭建改进的轻量化神经网络 SqueezeNet 模型,并用公开数据集进行初步验证和训练;}
            \item {仿真并对比FFT和同步压缩小波变化对数据特征增强的性能,最终选择FFT做数据变换;}
            \item {系统硬件电路设计与调试,主要是高速 ADC 采集电路和运算平台(FPGA+STM32F7);}
            \item {在 FPGA 中实现 ADC 采集和数据预处理操作,并将其处理后的数据发送至STM32;}
            \item {在 STM32 中部署轻量化 SqueezeNet 模型,分类准确率达到98\%。}
        \end{itemize}]
    [Python, C/C++, Keras, SequeezeNet, FPGA, Git]
    
%    \separator{0.5ex}
    
%    \experience%
%    [2021.1]
%    {2021.4}%
%    {\textbf{自学项目 @ Web Server }}%
%    [
%    学习《高性能服务器编程》及springsnail源码后,使用C++11编写的可处理高并发I/O请求的Web服务器。
%    \begin{itemize}
%    	\item {使用epoll边沿触发的IO多路复用技术,非阻塞IO,使用Reactor模式;}
%    	\item {使用多线程充分利用多核CPU,并使用线程池避免线程频繁创建、销毁的开销;}
%    	\item {使用状态机解析HTTP请求报文,支持解析GET和POST请求;}
%    	\item {实现同步/异步日志系统,记录服务器运行状态;}
%    \end{itemize}]
%    [C++, Linux, thread pool, Git]
    
\end{experiences}