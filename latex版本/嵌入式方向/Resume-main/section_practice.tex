%======================================================================
\sectionTitle{实习经历}{\faSuitcase}
%======================================================================
\begin{experiences}
	\experience%
	[2021.6]
	{至今}%
	{\textbf{腾讯科技(深圳)有限公司 @ TEG/云架构平台部/操作系统组/后台开发实习生}}%
	[
	通过搭建 Tencent OS 内核宕机查错的自动分析工具 t-crash 来快速定位引起内核崩溃的问题。
	\begin{itemize}
		\item {熟悉业务代码、开发测试流程以及测试环境,并负责 t-crash 中子模块 mce 的检测与代码编写;}
		\item {适配 t-crash 至 tlinux kernel 3,修复遗留BUG,并编写脚本对其进行压测;}
		\item {分析业务需求,编写完整后台程序,接收并解析服务器的宕机诊断信息,进行后续处理和入库,将诊断结果放至部门网站同时通过邮件发送给运维人员,目前已经在内部使用;}
		\item {根据业务特点,将CrashData、Diagnose、cmdbData、CrashDB、TapdData、MailData等主要流程封装成类,提高了代码的复用性以及后续新增子宕机问题的可扩展性;}
	\end{itemize}
	]
	[C, Python, Shell, Git, Redis, MySQL]
	
\end{experiences}